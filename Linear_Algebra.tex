\documentclass{article}
\usepackage{graphicx}
\usepackage{color}
\usepackage[letterpaper, margin=1in]{geometry}
\usepackage{amsmath}
\begin{document}
\setlength{\parindent}{0pt}

\textbf{Similarity Matrices}
\vspace{5mm}
Two matrices A and B or linear operators are called similar or conjugate, if there exists an invertible matrix s.t.

\begin{equation}
B=PAP^{-1}
\end{equation}

This means that they share many important properties:

\begin{enumerate}
\item \textbf{Rank} (dimension of the vector space generated by spanned by the columns/rows (they are equivalent). Easy way to get it is by row reduction. It is also a measurement of the non-degenerateness of a linear system of equations. 
\item \textbf{Characteristic polynomial} has eigen values as the roots and determinant and trace as the roots
\item eigenvalues
\item determinant
\item trace
\item \textbf{Minimal polynomial}
\item \textbf{Rational canonical form}
\item \textbf{Elementary divisors}


\end{enumerate}

Note that the eigenspace generated by two similar matrices which not be the same because the eigen space is transformed by the base change matrix used



\textbf{Jordan Forms}



\textbf{Normal matrix}


\textbf{Tensors}


\textbf{Hermitian Adjoint}

\end{document}